\documentclass[a4paper,
               %boxit,        % check whether paper is inside correct margins
               %titlepage,    % separate title page
               %refpage       % separate references
               %biblatex,     % biblatex is used
               keeplastbox,   % flushend option: not to un-indent last line in References
               %nospread,     % flushend option: do not fill with whitespace to balance columns
               %hyphens,      % allow \url to hyphenate at "-" (hyphens)
               %xetex,        % use XeLaTeX to process the file
               %luatex,       % use LuaLaTeX to process the file
               ]{jacow}
%
% ONLY FOR \footnote in table/tabular
%
\usepackage{pdfpages,multirow,ragged2e} %
%
% CHANGE SEQUENCE OF GRAPHICS EXTENSION TO BE EMBEDDED
% ----------------------------------------------------
% test for XeTeX where the sequence is by default eps-> pdf, jpg, png, pdf, ...
%    and the JACoW template provides JACpic2v3.eps and JACpic2v3.jpg which
%    might generates errors, therefore PNG and JPG first
%
\makeatletter%
	\ifboolexpr{bool{xetex}}
	 {\renewcommand{\Gin@extensions}{.pdf,%
	                    .png,.jpg,.bmp,.pict,.tif,.psd,.mac,.sga,.tga,.gif,%
	                    .eps,.ps,%
	                    }}{}
\makeatother

% CHECK FOR XeTeX/LuaTeX BEFORE DEFINING AN INPUT ENCODING
% --------------------------------------------------------
%   utf8  is default for XeTeX/LuaTeX
%   utf8  in LaTeX only realises a small portion of codes
%
\ifboolexpr{bool{xetex} or bool{luatex}} % test for XeTeX/LuaTeX
 {}                                      % input encoding is utf8 by default
 {\usepackage[utf8]{inputenc}}           % switch to utf8

\usepackage[USenglish]{babel}


%
% if BibLaTeX is used
%
\ifboolexpr{bool{jacowbiblatex}}%
 {%
  \addbibresource{jacow-test.bib}
  \addbibresource{biblatex-examples.bib}
 }{}
\listfiles

%%
%%   Lengths for the spaces in the title
%%   \setlength\titleblockstartskip{..}  %before title, default 3pt
%%   \setlength\titleblockmiddleskip{..} %between title + author, default 1em
%%   \setlength\titleblockendskip{..}    %afterauthor, default 1em

\begin{document}

\title{preparation OF papers for \NoCaseChange{JACoW} conferences\thanks{Work supported by ...}}

\author{A. N. Author\thanks{email address}, H. Coauthor, Name of Institute or Affiliation, [Postal Code] City, Country \\
		P. Contributor\textsuperscript{1}, Name of Institute or Affiliation, [Postal Code] City, Country \\
		\textsuperscript{1}also at Name of Secondary Institute or Affiliation, [Postal Code] City, Country}
	
\maketitle

%
\begin{abstract}
   Many conference series have adopted the same standards
   for electronic publication and have joined the Joint
   Accelerator Conferences Website (JACoW) collaboration
   for the publication of their proceedings. This document
   describes the common requirements for the submission of
   papers to these conferences. Please consult individual
   conference information for page limits, method of electronic
   submission, etc. It is not intended that this should
   be a tutorial in word processing; the aim is to explain the
   particular requirements for electronic publication at
   www.JACoW.org. The abstract itself is to act as a standalone
   entity and, as such, should not include citations. Any acronyms 
   should be expanded on their first occurrence, both in the 
   abstract and in the rest of the paper. 
   The abstract itself is to act as a standalone entity and, 
   as such, should not include citations. Any acronyms should 
   be expanded on their first occurrence, both in the abstract 
   and in the rest of the paper.
\end{abstract}


\section{SUBMISSION OF PAPERS}
Each author should submit the PDF file and all source
files (text and figures) to enable the paper to be
reconstructed if there are processing difficulties.

\section{MANUSCRIPTS}
Templates are provided for recommended software and
authors are advised to use them. Please consult the
individual conference help pages if questions arise.

\subsection{General Layout}

These instructions are a typical implementation of the
requirements. Manuscripts should have:
\begin{Itemize}
    \item  Either A4 (\SI{21.0}{cm}~$\times$~\SI{29.7}{cm}; \SI{8.27}{in}~$\times$~\SI{11.69}{in}) or US
           letter size (\SI{21.6}{cm}~$\times$~\SI{27.9}{cm}; \SI{8.5}{in}~$\times$~\SI{11.0}{in}) paper.
    \item  Single-spaced text in two columns of \SI{82.5}{mm} (\SI{3.25}{in}) with \SI{5.3}{mm}
           (\SI{0.2}{in}) separation. More recent versions of Microsoft Word have a default spacing of 1.5 lines;
           authors must change this to 1 line.
    \item  The text located within the margins specified in Table~\ref{l2ea4-t1}.
\end{Itemize}
\begin{table}[!hbt]
   \centering
   \caption{Margin Specifications}
   \begin{tabular}{lcc}
       \toprule
       \textbf{Margin} & \textbf{A4 Paper}                      & \textbf{US Letter Paper} \\
       \midrule
           Top         & \SI{37}{mm} (\SI{1.46}{in})            & \SI{0.75}{in} (\SI{19}{mm})        \\ %[3pt]
          Bottom       & \SI{19}{mm} (\SI{0.75}{in})            & \SI{0.75}{in} (\SI{19}{mm})        \\ %[3pt]
           Left        & \SI{20}{mm} (\SI{0.79}{in})            & \SI{0.79}{in} (\SI{20}{mm})        \\ %[3pt]
           Right       & \SI{20}{mm} (\SI{0.79}{in})            & \SI{1.02}{in} (\SI{26}{mm})        \\
       \bottomrule
   \end{tabular}
   \label{l2ea4-t1}
\end{table}

\subsection{Fonts}

In order to produce good Adobe Acrobat PDF files, authors
using the `jacow' \LaTeX{} template are asked to use only the fonts
defined in the ‘jacow’ class file (v2.2 of 2018/02/23) in standard, 
bold (i.\,e., \verb|\textbf|) or italic (i.\,e., \verb|\textit|) form and
symbols from the standard set of fonts. In Word use only
Symbol and, depending on your platform, Times or Times New Roman
fonts in standard, bold or italic form.

The layout of the text on the page is illustrated in
Fig.~\ref{fig:paper_layout}. Note that the paper’s title and the author list should
be the width of the full page. Tables and figures may span
the whole \SI{170}{mm} page width, if desired (see Fig.~\ref{fig:jacow_team}), but
if they span both columns, they should be placed at either
the top or bottom of a page to ensure proper flow of the
text (which should flow from top to bottom in each column 
and the numbering should be in sequence).

\begin{figure}[!htb]
   \centering
   \includegraphics*[width=.7\columnwidth]{JACpic_mc}
   \caption{Layout of papers.}
   \label{fig:paper_layout}
\end{figure}

\begin{figure*}[!tbh]
    \centering
    \includegraphics*[width=\textwidth]{jacow2017}

    \caption{Example of a full-width figure showing the JACoW Team at their annual
    	     meeting in November 2017. This figure has a multi-line caption that has to be
    	     justified rather than centred.}
    \label{fig:jacow_team}
\end{figure*}

\subsection{Title and Author List}

The title should use \SI{14}{pt} bold uppercase letters and be centred on the page.
Individual letters may be lowercased to avoid misinterpretation (e.\,g., mW, GeV, SPRing-8, SwissFEL).
To include a funding support statement, put an asterisk after the title and
the support text at the bottom of the first column on page~1---in Word,
use a text box; in \LaTeX, use $\backslash$\texttt{thanks}. See also the
subsection on footnotes.

The names of authors, their organizations/affiliations and
postal addresses should be grouped by affiliation and
listed in \SI{12}{pt} upper- and lowercase letters. The name of
the submitting or primary author should be first, followed
by the coauthors, alphabetically by affiliation. Where
authors have multiple affiliations, the secondary affiliation
may be indicated with a superscript, as shown in the
author listing of this paper. See \textbf{ANNEX~A} for further examples.

\subsection{Section Headings}

Section headings should not be numbered. They should
use  \SI{12}{pt}  bold  uppercase  letters  and  be  centred  in  the
column. All section headings should appear directly above
the text---there should never be a column break between a heading and the
following paragraph.

\subsection{Subsection Headings}

Subsection  headings  should  not  be  numbered.
They should use \SI{12}{pt} italic letters and be left aligned in the column.
Subsection headings use Title Case (or Initial Caps)
and should appear directly above the text---there should never be a column break
between a subheading and the following paragraph.

\subsubsection{Third-level Headings} These should use \SI{10}{pt} bold
letters and be run into the paragraph text. In \LaTeX{} these headings are
created with \LaTeX's \verb|\subsubsection| command.
In the Word templates authors must bold the heading text themselves.
This heading should be used sparingly. See Table~\ref{tab:styles}
for its style details.

\subsection{Paragraph Text}

Paragraphs should use \SI{10}{pt} font and be justified (touch each side) in
the column. The beginning of each paragraph should be indented
approximately \SI{0.33}{cm} (\SI{0.13}{in}). The last line of a paragraph should not be
printed by itself at the beginning of a column nor should the first line of
a paragraph be printed by itself at the end of a column.

\subsection{Figures, Tables and Equations}

Place figures and tables as close to their place of mention as
possible. Lettering in figures and tables should be large enough to
reproduce clearly. Use of non-approved fonts in figures can lead to
problems when the files are processed. \LaTeX{} users should be sure to use
non-bitmapped versions of Computer Modern fonts in equations (Type\,1 PostScript
or OpenType fonts are required. Their use is described in the help 
pages of the JACoW website\cite{jacow-help}).

Each figure and table must be numbered in ascending
order (1, 2, 3, etc.) throughout the paper. 
Figure captions are placed below figures, and table
captions are placed above tables.

Figure captions are formatted as shown in Figs.~\ref{fig:paper_layout} and \ref{fig:jacow_team},
while table captions take the form of a heading,
with initial letters of principle words, capitalized, and
without a period at the end (see Tables~\ref{eq:units} and \ref{tab:styles}).
Any reference to the contents of the table should be made from
the body of the paper rather than from within the table
caption itself.

Single-line captions are centred in the column, while captions
that span more than one line should be justified.
The \LaTeX{} template uses the ‘booktabs’ package to
format tables. 

When referring to a figure from within the text, the
convention is to use the abbreviated form [e.\,g., Fig.~1]
\emph{unless} the reference is at the start of the sentence, in
which case “Figure” is written in full. Reference to a
table, however, is never abbreviated [e.\,g., Table~1].


If a displayed equation needs a number (i.\,e., it will be
referenced), place it in parentheses, and flush with the
right margin of the column. The equation itself should be
indented and centred, as far as is possible:
\begin{equation}\label{eq:units}
    C_B=\frac{q^3}{3\epsilon_{0} mc}=\SI{3.54}{\micro eV/T}
\end{equation}

When referencing a numbered equation, use the word
“Equation” at the start of a sentence, and the abbreviated
form, “Eq.”, if in the text. The equation number is placed
in parentheses [e.g., Eq. (1)].

\subsection{Units}
	
Units should be written using the standard, roman font,
not the italic font, as shown in Eq.~(\ref{eq:units}).
An unbreakable space should precede a unit (in \LaTeX{} use a “\verb|\,|”,
the template uses the ‘siunitx’ package to format units).
Some examples are: \SI{3}{keV},
\SI{100}{kW}, \SI{7}{µm}. When a unit appears in a hyphenated,
compound adjective that precedes a noun, it takes on the
singular form, e.\,g., the 3.8-metre long undulator.

\subsection{References}
%
% References examples given here are not formatted using \cite as there
%			 are only reference numbers [1] and [2] in the template
%
All bibliographical and web references should be numbered and listed at the
end of the paper in a section called \textbf{REFERENCES}. When citing a
reference in the text, place the corresponding reference number in square
brackets~[1]. The reference citations in the text should be numbered
in ascending order. Multiple citations should appear in
the same bracket~[3, 4] and
with ranges where appropriate~[1--4, 10].

A URL or DOI may be included as part of a reference, but its
hyperlink should NOT be added. The usual practice is to
use a monospace font for the URL so as to help distinguish
it from normal text. In \LaTeX{} the ‘url’ package is used.

For authors to properly cite the resources used when researching
their papers is an obligation. In the interest of
promoting uniformity and complete citations, the IEEE
Editorial Style for Transactions and Journals, which itself 
adheres to the Chicago Manual of Style, has been
adopted~\cite{IEEE}. Please consult the appended material, \textbf{ANNEX~B},
for details. The onus is on authors to pay attention to
the details of the said style to ensure complete, accurate
and properly formatted references.

  \begin{table}[h!b]
	\setlength\tabcolsep{3.5pt}
	\centering
	\caption{Summary of Styles}
	\label{tab:styles}
	\begin{tabular}{llcc}
		\toprule
		\textbf{Style} & \textbf{Font}               & \textbf{Space}  & \textbf{Space} \\
		&                             & \textbf{Before} & \textbf{After} \\
		\midrule
		\textbf{PAPER}  & \SI{14}{pt}                 & \SI{0}{pt}      & \SI{3}{pt}  \\
		\textbf{TITLE}  & \textbf{UPPERCASE}          &                 &      \\
		& \textbf{EXCEPT FOR}         &                 &      \\
		& \textbf{REQUIRED lowercase} &                 &      \\
		& \textbf{letters}            &                 &      \\
		& \textbf{Bold}               &                 &      \\[5pt]
		%\midrule
		Author list  & \SI{12}{pt}                 & \SI{9}{pt}      & \SI{12}{pt} \\
		& UPPER- and lowercase        &                 &      \\[5pt]
		%\midrule
		\textit{Abstract} & \SI{12}{pt}                 & \SI{0}{pt}      & \SI{3}{pt} \\
		\textit{Title}  & \textit{Initial Caps}       &                 &      \\
		& \textit{Italic}             &                 &      \\[5pt]
		%\midrule
		\textbf{Section}  & \SI{12}{pt}                 & \SI{9}{pt}      & \SI{3}{pt}  \\
		\textbf{Heading}  & \textbf{UPPERCASE}          &                 &      \\
		& \textbf{bold}               &                 &      \\[5pt]
		%\midrule
		\textit{Subsection} & \SI{12}{pt}                 & \SI{6}{pt}      & \SI{3}{pt}  \\
		\textit{Heading}
                             & \textit{Initial Caps}       &                 &      \\
		& \textit{Italic}             &                 &      \\[5pt]
		%\midrule
		\textbf{Third-level} & \SI{10}{pt}                 & \SI{6}{pt}      & \SI{0}{pt}  \\
		\textbf{Heading}     
                            & \textbf{Initial Caps}       &                 &      \\
		& \textbf{Bold}               &                 &      \\[5pt]
		%\midrule
		Figure        & \SI{10}{pt}                 & \SI{3}{pt}      & $\ge$\SI{3}{pt}  \\
		Captions      &                             &                 &      \\[5pt]
		%\midrule
		Table         & \SI{10}{pt}                 & $\ge$\SI{3}{pt} & \SI{3}{pt}  \\
		Captions      &                             &                 &      \\[5pt]
		%\midrule
		Equations     & \SI{10}{pt} base font       & $\ge$\SI{6}{pt}     & $\ge$\SI{6}{pt} \\[5pt]
		%\midrule
		References      & \SI{9}{pt}				& \SI{0}{pt}      & \SI{3}{pt} \\
        when $\le9$ 	& \verb|\bibliography{9}|	&                 &  \\[5pt]
        Refs. $\ge10$ 	& \SI{9}{pt}				& \SI{0}{pt}      & \SI{3}{pt}  \\
                		& \verb|\bibliography{99}|	&    &    \\
		\bottomrule   %\SI{0.25}{in}
	\end{tabular}
\end{table}

\subsection{Footnotes}

Footnotes on the title and author lines may be used for acknowledgements
and e-mail addresses. A non-numeric sequence of characters (*, \#,
\dag, \ddag, \P) should be used to indicate the footnote.
These “pseudo footnotes” should only
appear at the bottom of the first column on the first page.

Any other footnote in the body of the paper should
use the normal numeric sequencing (i.\,e., 1, 2, 3)
and appear at the bottom of the same column in which
it is used.  All footnotes are of 8pt font size.

\subsection{Acronyms}

Acronyms should be defined the first time they appear, 
both in the abstract and in the rest of the paper. 

\section{STYLES}

Table~\ref{tab:styles} summarizes the fonts and spacing used in the styles of
a JACoW template. In \LaTeX, these 
are implemented in the ‘jacow’ class file.

\section{PAGE NUMBERS}

\textbf{DO NOT include any page numbers}. They will be added
when the final proceedings are produced.

\section{TEMPLATES}

Template documents for the recommended word processing
software are available from the JACoW website~\cite{jacow-help}
and exist for \LaTeX, Microsoft Word (Mac and PC)
and LibreOffice/Apache OpenOffice for US letter and A4
paper sizes. Use the correct template for your paper size and
platform.

Fonts are embedded by default with pdf\LaTeX. Using \LaTeX{} with `dvips', 
make sure that `ps2pdf' has the option \texttt{-dEmbedAllFonts=true}'.
Fonts of included figure graphics in PDF or EPS are often not embedded. 
So make sure that this done when generating them or reprocess them 
in `Ghostscript' with the switch \texttt{-dEmbedAllFonts=true}' set.

\flushcolsend

\section{CHECKLIST FOR ELECTRONIC PUBLICATION}
Authors are requested to go over the following checklist for electronic publication:
\begin{Itemize}
	\item  Use only Times or Times New Roman (standard, bold or italic) and Symbol
	fonts for text, \SI{10}{pt} except references, which should be \SI{9}{pt}.
	
	\item  Figures should use Times or Times New Roman (standard, bold or italic) and
	Symbol fonts when possible---\SI{6}{pt} minimum, with fonts embedded.
	\item  Check that citations to references appear in sequential order and
	that all references are cited.
	\item  Check that the PDF file prints correctly.
	\item  Check that there are no page numbers.
	\item  Check that the margins on the printed version are within \SI{\pm1}{mm}
	of the specifications.
	\item  \LaTeX{} users can check their margins by invoking the
	\texttt{boxit} option.
\end{Itemize}

Please also check the list of common oversights which can be found in \textbf{ANNEX C}.

\section{CONCLUSION}

Any conclusions should be in a separate section directly preceding
the \textbf{ACKNOWLEDGEMENTS}, \textbf{APPENDIX}, or \textbf{REFERENCES} sections, in that
order.

\section{ACKNOWLEDGEMENTS}
Any acknowledgement should be in a separate section directly preceding
the \textbf{REFERENCES} or \textbf{APPENDIX} section.


\section{APPENDIX}
Any appendix should be in a separate section directly preceding
the \textbf{REFERENCES} section. If there is no \textbf{REFERENCES} section,
this should be the last section of the paper.

%
% only for "biblatex"
%
\ifboolexpr{bool{jacowbiblatex}}%
	{\printbibliography}%
	{%
	% "biblatex" is not used, go the "manual" way
	
	%\begin{thebibliography}{99}   % Use for  10-99  references
	\begin{thebibliography}{9} % Use for 1-9 references
	
	\bibitem{jacow-help}
		JACoW,
		\url{http://www.jacow.org}
	
	\bibitem{IEEE}
		\textit{IEEE Editorial Style Manual},
		IEEE Periodicals, Piscataway,
		NJ, USA, Oct. 2014, pp. 34--52.
	\end{thebibliography}

} % end \ifboolexpr
%
% for use as JACoW template the inclusion of the ANNEX parts have been commented out
% to generate the complete documentation please remove the "%" of the next two commands
% 
%\newpage

%\include{annexes-A4}

\end{document}
