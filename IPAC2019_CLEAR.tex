\documentclass[a4paper,
               %boxit,        % check whether paper is inside correct margins
               %titlepage,    % separate title page
               %refpage       % separate references
               %biblatex,     % biblatex is used
               keeplastbox,   % flushend option: not to un-indent last line in References
               %nospread,     % flushend option: do not fill with whitespace to balance columns
               %hyphens,      % allow \url to hyphenate at "-" (hyphens)
               %xetex,        % use XeLaTeX to process the file
               %luatex,       % use LuaLaTeX to process the file
               ]{jacow}
%
% ONLY FOR \footnote in table/tabular
%
\usepackage{pdfpages,multirow,ragged2e} %
%
% CHANGE SEQUENCE OF GRAPHICS EXTENSION TO BE EMBEDDED
% ----------------------------------------------------
% test for XeTeX where the sequence is by default eps-> pdf, jpg, png, pdf, ...
%    and the JACoW template provides JACpic2v3.eps and JACpic2v3.jpg which
%    might generates errors, therefore PNG and JPG first
%
\makeatletter%
\ifboolexpr{bool{xetex}}{\renewcommand{\Gin@extensions}{
    .pdf,%
    .png,.jpg,.bmp,.pict,.tif,.psd,.mac,.sga,.tga,.gif,%
    .eps,.ps,%
  }}{
}
\makeatother

% CHECK FOR XeTeX/LuaTeX BEFORE DEFINING AN INPUT ENCODING
% --------------------------------------------------------
%   utf8  is default for XeTeX/LuaTeX
%   utf8  in LaTeX only realises a small portion of codes
%
\ifboolexpr{bool{xetex} or bool{luatex}} % test for XeTeX/LuaTeX
 {}                                      % input encoding is utf8 by default
 {\usepackage[utf8]{inputenc}}           % switch to utf8

\usepackage[USenglish]{babel}


%
% if BibLaTeX is used
%
\ifboolexpr{bool{jacowbiblatex}}%
 {%
  \addbibresource{jacow-test.bib}
  \addbibresource{biblatex-examples.bib}
 }{}
\listfiles

%%
%%   Lengths for the spaces in the title
%%   \setlength\titleblockstartskip{..}  %before title, default 3pt
%%   \setlength\titleblockmiddleskip{..} %between title + author, default 1em
%%   \setlength\titleblockendskip{..}    %afterauthor, default 1em

\usepackage{todonotes}
\usepackage{caption, subcaption}
\usepackage{tikz}
\usepackage{calc}

\newcommand{\correctorMagnet}[2]{
    \filldraw[blue!80] (#1,0) -- (#1-0.15, 0.7) -- (#1+0.15, 0.7);
    \filldraw[blue!80] (#1,0) -- (#1-0.15,-0.7) -- (#1+0.15,-0.7);
    
    \node[rotate=-90,anchor=west] at (#1,-0.7) {#2};
}

\newcommand{\BTV}[2]{
    \draw[ultra thick, purple] (#1,0) circle (0.4);
    \draw[ultra thick, purple] (#1,-0.3) -- (#1,0.3);
    
    \node[rotate=-90,anchor=west] at (#1,-0.7) {#2};
}

\newcommand{\cBPM}[2]{
    \draw[ultra thick, purple] (#1,0) circle (0.15);
    
    \node[rotate=-90,anchor=west] at (#1,-0.7) {#2};
}
\newcommand{\iBPM}[2]{
    \draw[ultra thick, purple] ({#1-0.25},-0.25) rectangle ({#1+0.25},+0.25);
    
    \node[rotate=-90,anchor=west] at (#1,-0.7) {#2};
}

\newcommand{\lensF}[2] {
    \pgfmathsetmacro{\lensRadius}{2};
    \pgfmathsetmacro{\lensHeight}{0.7};
    \pgfmathsetmacro{\startAngle}{asin(\lensHeight/\lensRadius)};
    
    \draw [fill=blue!15]  (#1,\lensHeight)
         arc[start angle=180-\startAngle,delta angle=2*\startAngle,radius=\lensRadius]
         arc[start angle=-\startAngle,delta angle=2*\startAngle,radius=\lensRadius]
         -- cycle; % to get a better line end

   \node[rotate=-90,anchor=west] at (#1,-0.7) {#2};
}

\newcommand{\lensD}[2] {
    \pgfmathsetmacro{\lensRadius}{2};
    \pgfmathsetmacro{\lensHeight}{0.7};
    \pgfmathsetmacro{\startAngle}{asin(\lensHeight/\lensRadius)};
    
    \draw [fill=blue!15]  (#1,\lensHeight) --
         (#1-0.2,\lensHeight)
         arc[start angle= 180+\startAngle,
             end angle  = 180-\startAngle,
             radius     = -\lensRadius] --
         (#1+0.2,-\lensHeight)
         arc[start angle=\startAngle,
             end angle=180-\startAngle,radius=\lensRadius]
         %.-- cycle; % to get a better line end

   \node[rotate=-90,anchor=west] at (#1,-0.7) {#2};
}

% SEE ALSO: https://hackmd.web.cern.ch/MXDrpL8NSjKBrgtns_t7Ig#

\begin{document}

\title{Status of the CLEAR electron beam user facility at CERN}

\author{K. N. Sjobak\thanks{k.n.sjobak@fys.uio.no}\textsuperscript{1}, E. Adli, C. A. Lindstrom, University of Oslo, Oslo, Norway\\
  M. Bergamaschi, S. Burger, R. Corsini, A. Curcio, S. Curt, S. Doebert, W. Farabolini, D. Gamba,\\
  L. Garolfi, A. Gilardi, I. Gorgisyan, E. Granados, H. Guerin, R. Kieffer, M. Krupa, T. Lefevre,\\
  S. Mazzoni, G. McMonagle, J. Nadenau, H. Panuganti, S. Pitman, V. Rude, A. Schlogelhofer,\\
  P. K. Skowronski, M. Wendt, A. Zemanek, CERN, Geneva, Switzerland \\
  A. Lyapin, UCL, London, United Kingdom \\
  \textsuperscript{1}also at CERN, Geneva, Switzerland}

\maketitle

%
\begin{abstract}
  The CERN Linear Electron Accelerator for Research (CLEAR) is now in its second year of operation, providing a testbed for new accelerator technologies and a versatile radiation source.
  Hosting a varied experimental program, this beamline provides a flexible test facility for users both internal and external to CERN, as well as being an excellent training ground for young accelerator physicists.
  The energy can be varied between 60 and 220 MeV, bunch length between 1 and 3 ps, bunch charge in the range 1 to 200 pC, and number of bunches in the range 1 to 200, at a repetition rate of 0.8 to 10 Hz.
  X-band test capabilities have been added to the facility during this run, further enriching the range of accelerator R\&D studies performed at CLEAR.
  The status of the facility with an overview of the recent experimental results will be presented.
\end{abstract}

\section{Introduction}
\todo[inline]{
%\textbf{(Kyrre, Veronica, Riccardo)}\\
 * Describe what CLEAR is and brief history\\
 * Reference earlier CLEAR machine papers\\
 * Overview picture of the beam line
}

\begin{figure*}
    \centering
    \begin{subfigure}[h]{\textwidth}
        \centering
        %% TIKZ DRAWING OF CALIFES ELEMENTS
%% TO BE INCLUDED INTO A LATEX DOCUMENT
%% K. Sjobak, 2018

\begin{tikzpicture}

    %EOS table
    \filldraw[green] (4.9,-0.4) rectangle (2.9,0.4);
    \filldraw[green] (4.9,-0.4) rectangle (4.7,-2);
    \filldraw[green] (2.9,-0.4) rectangle (3.1,-2);
    
    %BEAM
    \draw[latex-,ultra thick] (0,0)--(15,0);
    \draw[ultra thick, dashed] (-1,0) -- (0,0);
    
    %% GUN
    \node at (15,1) {\belemsiz 0.0~m};
    
    % Gun solenoids
    \solRect{15}{0.5}{14}{0.7}
    \solRect{15}{-0.5}{14}{-0.7}
    
    \solRect{15.25}{0.5}{15}{0.7}
    \solRect{15.25}{-0.5}{15}{-0.7}
    
    % Gun cavity
    \draw[orange, ultra thick] (15,0.0) to (15,0.4)
        arc(90:180:0.15)
        arc(360:180:0.05) arc(0:180:0.15)
        arc(360:180:0.05) arc(0:180:0.15);
    \draw[orange, ultra thick] (15,0.0) to (15,-0.4)
        arc(-90:-180:0.15)
        arc(0:180:0.05) arc(0:-180:0.15)
        arc(0:180:0.05) arc(0:-180:0.15);
    
    \node[rotate=-90,anchor=west,align=left] at (15+0.25/2, -0.7)
        {\belemsiz SNH 110};
    \node[rotate=-90,anchor=west,align=left] at (14.5, -0.7)
        {\belemsiz GUN 115\\
         \belemsiz SNI 120};
    
    \correctorMagnet{13.7}{DG 130};
    \node at (13.7,1) {\belemsiz 0.32~m};
    
    %Laser mirror
    \draw[ultra thick]  ({(13.7-13)/2+13 + 0.1}, -0.1) --
                        ({(13.7-13)/2+13 - 0.1}, -0.3);
    
    %Laser
    \draw[thick,-latex,dashed]
        ({(13.7-13)/2+13},-2.25) --
        ({(13.7-13)/2+13},-0.2) -- 
        (15,0);
        
    %Laser table
    \BTV[-2.25]{14}{};
    \draw[thick, -latex, dashed]
        (13,-2.25) -- (14,-2.25);
    \node at (14.9,-2.25) {\belemsiz BTV125};
    
    \draw[ultra thick] ({(13.7-13)/2+13-0.25/2}, -2.25-0.25/2) -- 
                       ({(13.7-13)/2+13+0.25/2}, -2.25+0.25/2);
    
    %ICT 210
    \filldraw[purple] (12.95-0.1/2,-0.3) rectangle (12.95+0.1/2,0.3);
    \node[rotate=-90,anchor=west] at (12.95,-0.7) {\belemsiz ICT 210};

    \BTV{12.35}{MTV 215};
    \node at (12.35,0.7) {\belemsiz 1.81~m};
    
    \cBPM{11.8}{BPC 220}{0.15};
    \correctorMagnet{11.5}{DG 225};
    \node at (11.5,1) {\belemsiz 2.18~m};
    
    \draw[ultra thick, orange] (11.3,-0.4) rectangle (10.2,0.4);
    
    \solRect{11.3}{0.6}{10.2}{0.7}
    \solRect{11.3}{-0.6}{10.2}{-0.7}
    
    \filldraw[blue] (11.3, 0.45) rectangle (10.2,  0.55);
    \filldraw[blue] (11.3,-0.45) rectangle (10.2, -0.55);
    
    \node[rotate=-90,anchor=west, align=left] at (10.2+1.1/2,-0.7)
        {\belemsiz ACS 230\\
         \belemsiz DB 230-S\\
         \belemsiz SNG 230};
    
    \cBPM{9.9}{BPC 240}{0.15};
    \correctorMagnet{9.6}{DG 245};
    \node at (9.6,1) {\belemsiz 7.33~m};
    
    \draw[ultra thick, orange] (9.3,-0.4) rectangle (8.2,0.4);
    \node[rotate=-90,anchor=west, align=left] at (8.2+1.1/2,-0.7) {\belemsiz ACS 250\\
     \belemsiz SNG 250};

    \solRect{9.3}{0.5}{8.2}{0.7};
    \solRect{9.3}{-0.5}{8.2}{-0.7};
    
    \cBPM{7.9}{BPC 260}{0.15};
    \correctorMagnet{7.6}{DG 265};
    \node at (7.6,1) {\belemsiz 12.49~m};
    
    \draw[ultra thick, orange] (7.3,-0.4) rectangle (6.2,0.4);
    \node[rotate=-90,anchor=west] at (6.2+1.1/2,-0.7) {\belemsiz ACS 270};
    
    \cBPM{5.9}{BPC 310}{0.15};
    \correctorMagnet{5.6}{DG 320};
    \node at (5.6,1) {\belemsiz 17.70~m};
    
    \kickerHV[0]{5.125}{SDH 340}{-1}{orange};

    %\draw[ultra thick, purple] (4.15,-0.15) rectangle (3.85,0.15);
    
    \lensF{4.4}{QFD 350};
    \lensD{3.9}{QDD 355};
    \lensF{3.4}{QFD 360};
    \node at (3.9,1) {\belemsiz 18.90~m};

    
    %EOS table defined on top, to get behind the "beam"
    \draw[-latex, dashed,thick] (4.8,-2) -- (4.8,-0.2) -- (3.0,-0.2) -- (3,-2);
    %\node at (4.8,-2.25) {\belemsiz EOS laser};
    \node at (3.9,-2.25) {\textbf{Electro-Optical Sampling}};
    
    \cBPM{2.7}{BPC 380}{0.15};
    \correctorMagnet{2.4}{DG 385};
    \node at (2.4,1) {\belemsiz 19.96~m};

    \BTV{2.0}{MTV 390};
    
    %% VESPER
    
    \pgfmathsetmacro{\VesperStart}{1.5}; %Center of the dipole (X)
    \pgfmathsetmacro{\VesperAngle}{-22.0};
    %Arrow end point
    \pgfmathsetmacro{\VesperX}{-1};
    \pgfmathsetmacro{\VesperY}{(\VesperStart-\VesperX)*tan(\VesperAngle)};
    
    \pgfmathsetmacro{\VesperTabX}{-0.0};
    \pgfmathsetmacro{\VesperTabY}{(\VesperStart-\VesperTabX)*tan(\VesperAngle)};
    \filldraw[green,rotate around={-\VesperAngle:(\VesperTabX,\VesperTabY)}]
    (\VesperTabX-0.5, \VesperTabY-0.4) rectangle
    (\VesperTabX+0.5, \VesperTabY+0.4);
    %\node[rotate=-90,anchor=west] at (\VesperTabX,\VesperTabY-0.7) {\belemsiz VESPER};
    \node at (\VesperTabX+0.4, -2.5) {\textbf{VESPER}};
    
    \filldraw[gray,rotate around={-\VesperAngle:(\VesperX,\VesperY)}]
        (\VesperX-0.05, \VesperY-0.5) rectangle
        (\VesperX+0.05, \VesperY+0.5);
    
    \draw[->,ultra thick] (\VesperStart,0)--(\VesperX,\VesperY);
    
    \kickerHV[0]{\VesperStart}{BHB 400}{1}{blue};
    
    \pgfmathsetmacro{\VesperBTVaX}{0.9};
    \pgfmathsetmacro{\VesperBTVaY}{(\VesperStart-\VesperBTVaX)*tan(\VesperAngle)};
    \BTV[\VesperBTVaY]{\VesperBTVaX}{BTV420}

    % ICT 430
    \pgfmathsetmacro{\VesperICTX}{0.4};
    \pgfmathsetmacro{\VesperICTY}{(\VesperStart-\VesperICTX)*tan(\VesperAngle)};
    \filldraw[purple,rotate around={-\VesperAngle:(\VesperICTX,\VesperICTY)}]
        (\VesperICTX-0.1/2, \VesperICTY-0.3) rectangle
        (\VesperICTX+0.1/2, \VesperICTY+0.3);

    \node[rotate=-90,anchor=west] at (\VesperICTX,\VesperICTY-0.7) {\belemsiz ICT 430};

    \pgfmathsetmacro{\VesperBTVbX}{-0.5};
    \pgfmathsetmacro{\VesperBTVbY}{(\VesperStart-\VesperBTVbX)*tan(\VesperAngle)};
    \BTV[\VesperBTVbY]{\VesperBTVbX}{BTV440}
    
    
\end{tikzpicture}
        \caption{CALIFES injector}
    \end{subfigure}
    \begin{subfigure}[h]{\textwidth}
        \centering
        %% TIKZ DRAWING OF CLEAR EXPERIMENTAL BEAMLINE 1 ELEMENTS
%% TO BE INCLUDED INTO A LATEX DOCUMENT
%% K. Sjobak, 2018

\begin{tikzpicture}
    %BEAM INSTRUMENTATION TEST STAND AREA
    \filldraw[green] (11.3,-0.4) rectangle (13.7,0.4);
    \node at ({(11.3+13.7)/2},-2.25) {\textbf{BI test area}};

    %CLIC test area
    \filldraw[green] (10.9,-0.5) rectangle (6.55,0.5);
    \node at ({(10.3+7.8)/2},-2.25) {\textbf{CLIC test area}};

    %IN-AIR table
    \filldraw[green] (-0.6,-0.5) rectangle (0.5,0.5);
    \node[align=left,anchor=west] at (-0.6,1.0) {\textbf{In-Air}\\ \textbf{test area}};
    \filldraw[gray] (-0.6,-2.0) rectangle (-0.5,0.5);

    \draw[latex-,ultra thick] (0,0)--(16,0);

    \kickerHV[0]{15.8}{BHB 400}{1}{blue};

    \lensF{15.4}{QFD 510};
    \lensD{15.0}{QDD 515};
    \lensF{14.6}{QFD 520};
    \node at (15.0,1) {\belemsiz 22.86~m};

    \iBPM{14.1}{BPM 530};

    \correctorMagnet{13.8}{DJ 540};
    \node at (13.8,1) {\belemsiz 23.76~m};

    \BTV{13.3}{(BTV 545)};

    \filldraw[purple] (12.7+0.15,0.15) rectangle (12.7-0.15,-0.15);
    \node[rotate=-90,anchor=west] at (12.7,-0.7)
        {\belemsiz WCM 550};

    \correctorMagnet{12.4}{DJ 590};
    \node at (12.4,1) {\belemsiz 25.00~m};

    \iBPM{12.0} {BPM 595}; %{BPM 560};

    \filldraw[purple] (11.5+0.15,0.15) rectangle (11.5-0.15,-0.15);
    \node[rotate=-90,anchor=west] at (11.5,-0.7)
        {\belemsiz BPR 600};

    %\cBPM{11.1}{BPC 565}{0.15};
    \cBPM{11.1}{BPC 610}{0.15};

    \BTV{10.5}{BTV 620};
    \node at (10.5,1) {\belemsiz 25.85~m};

    \draw[ultra thick, orange] (10.1,-0.4) rectangle (9.7,0.4);
    \node[rotate=-90,anchor=west] at (9.9,-0.7)
        {\belemsiz ACS 640};

    \draw[ultra thick, orange] (9.6,-0.4) rectangle (9.2,0.4);
    \filldraw[purple] (9.55,-0.5) rectangle (9.5,0.5);
    \node[rotate=-90,anchor=west,align=left] at (9.4,-0.7)
        {\belemsiz WFM 645\\[-0.5em]
         \belemsiz ACS 650};

    \cBPM{9.0}{BPC 660}{0.1};
    \cBPM{8.7}{BPC 670}{0.1};
    \cBPM{8.4}{BPC 680}{0.1};
    \cBPM{8.1}{BPC 690}{0.1};

    \correctorMagnet{7.8}{DJ 710};
    \node at (7.8,1) {\belemsiz 29.32~m};

    \cBPM{7.5}{BPC 720}{0.15};

    \BTV{6.9}{BTV 730};

    \lensF{6.3}{QFD 760};
    \lensD{5.9}{QDD 765};
    \lensF{5.5}{QFD 770};
    \node at (5.9,0.9) {\belemsiz 30.62~m};


    \correctorMagnet{5.1}{DJ 780};
    \node at (5.1,1.15) {\belemsiz 31.66~m};

    \cBPM{4.8}{BPC 790}{0.15};

    \lensF[green]{4.3}{PLC 800\\[-0.5em]
             \belemsiz BTV 800\\[-0.5em]
             \belemsiz BTV 805};
    \node at (4.3,-2.25) {\textbf{Plasma lens}};
    \node at (4.3,0.9) {\belemsiz 32.26~m};


    \BTV{3.8}{BTV 810};

    \iBPM{3.2}{BPM 820};

    \correctorMagnet{2.8}{DJ 840};
    \node at (2.8,1) {\belemsiz 33.38~m};

    \lensD{2.3}{QDD 870};
    \lensF{1.9}{QFD 880};

    %% IN-AIR SPECTROMETER

    \pgfmathsetmacro{\InAirStart}{1.5};
    \pgfmathsetmacro{\InAirAngle}{-22.0};
    \pgfmathsetmacro{\InAirY}{(\InAirStart+0.6)*tan(\InAirAngle)};

    \draw[-latex,ultra thick] (\InAirStart,0)--(-0.6,\InAirY);
    \kickerHV[0]{\InAirStart}{BHB 900}{1}{blue};
    \node at (\InAirStart,1) {\belemsiz 35.03~m};


    \pgfmathsetmacro{\InAirBTVaX}{0.0};
    \pgfmathsetmacro{\InAirBTVaY}{(\InAirStart-\InAirBTVaX)*tan(\InAirAngle)};
    \BTV[\InAirBTVaY]{\InAirBTVaX}{BTV930};

    \pgfmathsetmacro{\InAirBPMx}{0.5};
    \pgfmathsetmacro{\InAirBPMy}{(\InAirStart-\InAirBPMx)*tan(\InAirAngle)};
    \iBPM[\InAirBPMy]{\InAirBPMx}{BPM920};

    %% IN-AIR TABLE INSTRUMENTATION
    \BTV{0.9}{BTV 910};
%    \node[anchor=south] at (0.9,0.3) {\belemsiz BTV 910};

    %ICT 210
    \filldraw[purple] (0.5-0.1/2,-0.3) rectangle (0.5+0.1/2,0.3);
    \node[anchor=south] at (0.5,0.3) {\belemsiz ICT 915};

\end{tikzpicture}
        \caption{Experimental beamline}
    \end{subfigure}
    \caption{Overview of the beamline elements at CLEAR}
\end{figure*}

\section{Operation and Performance}
\todo[inline]{
%\textbf{(Veronica, Kyrre)} \\
 * Improvements and changes since last year\\
 * Inductive BPM consolidation\\
 * Camera system\\
 * Laser system\\
 * New parameter space within reach?
}

\section{Connection of Xbox-1 to Xband structure}
\todo[inline]{
\textbf{(Veronica, Helmut)}\\
 * Picture, impact on operation\\
 * Reference Sam's paper for details
}

\section{Overview of experimental program}

\subsection{Medical irradiation tests}

\subsection{Terrahertz}

\subsection{CLIC module tests}
\todo[inline]{Kicks, WFMs, BPMs, tests with power, \ldots}
\subsection{Plasma lens}

\section{Future plans}
\todo[inline]{
%\textbf{(Veronica)}\\
 * Possibility of new beamline
}

\section{Conclusion}

\end{document}
